\documentclass{article}
	\usepackage{natbib}

\begin{document}

\title{Simulating International Energy Security}
	\author{David Masad}
	\maketitle

\section{Introduction}

Energy security is defined by the International Energy Agency as ``the uninterrupted availability of energy sources at an affordable price''\citep{iea_2013}. As the world's energy demand continues to expand in the face of growing scarcity, energy security is becoming increasingly critical not just to developed countries but to developing ones as well \citep{yergin_2006}. Thus, assessing the energy security of different countries is important in order to understand the overall state of the international system. Despite this, there have been few published attempts to create formal models of energy security. 

One such model, MOSES (Model of Short-term Energy Security), was created within the International Energy Agency (IEA), the intergovernmental organization of the world's major oil importers created as a counterbalance to exporter cartel OPEC \citep{}. As the name indicates, MOSES focuses on short-term risks, defined as a timescale of days and weeks. Furthermore, MOSES assigns risk scores based on an impressive, but ultimately static, set of indicators. In particular, the indicator used to measure political stability is ``weighted average of political stability of suppliers''. However, the greater the diversity of suppliers (another input into the model), the greater the potential range of political stabilities. Furthermore, averaging ratings of political risk eliminates the complexity that characterises international political stabilitiy, ignoring the possibility of crisis and conflict contagion \citep{}. 

Other models of energy security focus on long-term timeframes, in years and decades. [[EXPAND MORE HERE]]

In this paper, I present a novel model linking energy security with international political stability.

\section{Model Description}

\subsection{Overview}

The model consists of countries, each of which has a set supply of and demand for crude oil. Countries are linked by a network of export-import relationships. Each month, some countries enter crisis with fixed probabilities; the exports and imports of countries in crisis are considered \emph{at risk}. If \textbf{contagion} is enabled, a crisis in one country may spread to its immediate geographic neighbors. If \textbf{assistance} is enabled, oil exporters may increase their production to balance a loss of secure oil by their major import partners.

The model parameters currently remain fixed for each run: political instability, supply, demand, and trade relations are treated as constant. 

\subsection{Data Sources}
The model inputs are empirical whenever possible. Data inputs are: oil trade relationships, country political instability, and country consumption of domestic oil production. Total supply and demand are derived largely from input data.

The network of oil imports and exports comes from the United Nations-maintained COMTRADE database \citep{un_2013}. I extract all imports of crude petroleum (product code 2709) and other unprocessed petrolem (product code 2710) for 2012, the most recent year for which full data is available. Note that import data appears to be more reliable than export data: many major oil producers (particularly Saudi Arabia and Venezuela) only report export volume by region (e.g. Europe, North America); however, the majority of their trade partners report their imports from them.

Political instability estimates are taken from the Economist Intelligence Unit's Political Index ratings \citep{eiu_2013}. These are in turn estimated based on the methodology developed by the Political Instability Task Force \citep{goldstone_2005}, which could predict the onset of internal instability in a two-year period with 80\% accuracy. Thus, these scores are normalized such that a rating of 10 (the maximum) is associated with an 80\% probability of instability within a 24-month timeframe, translating into a [[FILLIN]] monthly probability of crisis onset.

For the majority of countries, I assume that supply and demand are equal to total export and imports, respectively. For the top ten oil producers, I use US Energy Information Administration (EIA) estimates \citep{} for those countries' consumption of their own domestic production. 

\subsection{Formal Description}

The model consists of \textbf{Country} agents, linked by two networks: a directed \textbf{trade network}, and an undirected \textbf{geographic adjacency} network, where the country agents are the network nodes.

Model variables may be \emph{data-derived}, \emph{parameters} that are fixed for each run and may be changed between them, or \emph{states} of the agents and edges that vary during the model. 

\subsubsection{Agent Description}

\section{Model Analysis}

\section{Discussion}



In this poster, I will present an initial implementation of a novel agent-based model linking crude oil energy security to inter- and intra-state crises and conflicts, through the mechanism of supply shocks \citep{kilian_2008}. The model operates at the meso (months-years) timescale, and allows countries to react to changes in the oil supply. It can generate ranges of notional near-future trajectories of global and local energy security, providing risk estimates for different countries. The model attempts to capture the emergent consequences of conflict contagion \citep{black_2013}, temporary production increases and varying levels of stockpiles. It can also be used to explore different scenarios, from particular crises to realignments of major trade relationships. 

The agents in the model are geospatially located countries, acting in monthly timesteps. Countries experience exogenous crises based on probabilities derived from empirical estimates (such as \citep{goldstone_2005,eiu_2013,ward_2013}). They are linked geographically, and by oil trade relationships obtained via the COMTRADE database \citep{un_2013}. As countries experience crises, their participation in the  international oil system is placed at risk, changing the volume of `secure' imports or exports of other countries and  potentially triggering supply or demand shocks. 

Countries respond to the changing energy security environment in several ways. Exporters may choose to temporarily increase production in response to supply shocks elsewhere, while countries that are members of the International Energy Agency may vote to release their emergency stockpiles. The model's behavior is verified against previous qualitative and quantitative analysis, and in consultation with subject-matter experts. 

The poster will describe the model's methodology and input data, and provide a visual representation of the model's estimated energy security risk for every country. It will provide a visual example of a representative model run, and illustrate the results of a sweep of the model's behavior and parameter space. Finally, it will describe future work, incorporating different energy sources, shipping chokepoints, and dynamic network updating.

 \bibliographystyle{chicago}
\bibliography{energysecurity}

\end{document}