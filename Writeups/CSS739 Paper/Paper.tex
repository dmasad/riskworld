\documentclass{article}
	\usepackage{natbib}
	\usepackage{mathtools}
	\usepackage{amsmath}
	\usepackage{algorithm2e}

\begin{document}

\title{Simulating International Energy Security}
	\author{David Masad}
	\maketitle

\section{Introduction}

Energy security is defined by the International Energy Agency as ``the uninterrupted availability of energy sources at an affordable price''\citep{iea_2013}. As the world's energy demand continues to expand in the face of growing scarcity, energy security is becoming increasingly critical not just to developed countries but to developing ones as well \citep{yergin_2006}. Thus, assessing the energy security of different countries is important in order to understand the overall state of the international system. Despite this, there have been few published attempts to create formal models of energy security. 

One such model, MOSES (Model of Short-term Energy Security), was created within the International Energy Agency (IEA), the intergovernmental organization of the world's major oil importers created as a counterbalance to exporter cartel OPEC \citep{}. As the name indicates, MOSES focuses on short-term risks, defined as a timescale of days and weeks. Furthermore, MOSES assigns risk scores based on an impressive, but ultimately static, set of indicators. In particular, the indicator used to measure political stability is ``weighted average of political stability of suppliers''. However, the greater the diversity of suppliers (another input into the model), the greater the potential range of political stabilities. Furthermore, averaging ratings of political risk eliminates the complexity that characterises international political stabilitiy, ignoring the possibility of crisis and conflict contagion \citep{}. 

Other models of energy security focus on long-term timeframes, in years and decades. [[EXPAND MORE HERE]]

In this paper, I present a novel model linking energy security with international political stability.

\section{Model Description}

\subsection{Overview}

The model consists of countries, each of which has a set supply of and demand for crude oil. Countries are linked geographically, and by a network of export-import relationships. Each month, some countries enter crisis with fixed probabilities; the exports and imports of countries in crisis are considered \emph{at risk}. If \textbf{contagion} is enabled, a crisis in one country may spread to its immediate geographic neighbors. If \textbf{assistance} is enabled, oil exporters may increase their production to balance a loss of secure oil by their major import partners.

The model parameters currently remain fixed for each run: political instability, supply, demand, and trade relations are treated as constant. 

\subsection{Data Sources}
The model inputs are empirical whenever possible. Data inputs are: oil trade relationships, country political instability, and country consumption of domestic oil production. Total supply and demand are derived largely from input data.

The network of oil imports and exports comes from the United Nations-maintained COMTRADE database \citep{un_2013}. I extract all imports of crude petroleum (product code 2709) and other unprocessed petrolem (product code 2710) for 2012, the most recent year for which full data is available. Note that import data appears to be more reliable than export data: many major oil producers (particularly Saudi Arabia and Venezuela) only report export volume by region (e.g. Europe, North America); however, the majority of their trade partners report their imports from them.

For the majority of countries, I assume that supply and demand are equal to total export and imports, respectively. For the top ten oil producers, I use US Energy Information Administration (EIA) estimates \citep{} for those countries' consumption of their own domestic production. 

Political instability estimates are taken from the Economist Intelligence Unit's Political Index ratings \citep{eiu_2013}. These are in turn estimated based on the methodology developed by the Political Instability Task Force \citep{goldstone_2005}, which could predict the onset of internal instability in a two-year period with 80\% accuracy. Thus, these scores are normalized such that a rating of 10 (the maximum) is associated with an 80\% probability of instability within a 24-month timeframe, translating into a [[FILLIN]] monthly probability of crisis onset.

Crisis duration is drawn independently from onset. \citet{cioffi_2004} and others have argued that internal and external conflict durations follow a power law distribution, which I use here. Specifically, in order to accomodate the limitations of the MASON random-number generator, I use a uniform approximation \citep{weisstein_2013}, defined as:
$$
\left((x_{1}^{\alpha} - x_{0}^{\alpha})y + x_{0}^{\alpha}\right)^{\frac{1}{\alpha}} \sim \text{Power Law}
$$
where 
$$Y \sim U[0,1]$$
$$ x_0  \equiv \text{minimum}$$
$$ x_1  \equiv \text{maximum}$$
$$ \alpha \equiv \text{Coefficient}$$

I calibrate the parameters from two datasets: \citet{PRIO} for armed conflict, and \citep{hendrix_2013}, the Social Conflict in Africa Database, which includes both armed and lower-level social conflicts. I subset the PRIO dataset for conflicts in Africa and merge it with SCAD, eliminating duplicates and events below the severity of a general strike, in order to obtain as complete as possible a set of crises and conflicts that may be associated with an oil production  shock. I find the duration of each event, and fit a power law to the resulting distribution, resulting in an estimated coefficient of $\mathbf{-1.37}$.

\subsection{Formal Description}

The model consists of \textbf{Country} agents, posessing several characteristics and linked by two networks: a directed \textbf{trade network}, and an undirected \textbf{geographic adjacency} network, where the country agents are the network nodes.

Model variables may be \emph{fixed} based on data, \emph{parameters} that are fixed for each run and may be changed between them, or \emph{states} of the agents and edges that vary during the model.

\subsubsection{Networks}

Agents are connected by two networks, which we can also express as matrices. Let $\mathbf{G}$ be the geographic adjacency matrix, such that $g_{i,j}=g_{j,i}=1$ when counties $i$ and $j$ are geographically contiguous or near-contiguous, and $0$ otherwise. 

Let $\mathbf{E}$ be the trade matrix, such that $e_{i,j}$ denotes the volume of oil exports from $i$ to $j$. Note that if the assistance submodel is turned on, the values of $\mathbf{E}$ may change over time.

\subsubsection{Agent parameters and variables}

Each agent represents an independent country. Agents are associated with the \textbf{geometry} of their position on a world map, which determines their neighbors. They are described by two submodels: an oil model, and a crisis model.

\paragraph{Oil Submodel}
\begin{itemize}
	\item \textbf{Domestic Share} of the country's oil demand satisfied by domestic production. For most countries, this defaults to 0; for the top ten oil producers, it is set based on US Energy Information Administration data \citep{}.
	\item \textbf{Total Imports} is the sum of the current volume of all oil imports. 
	\item \textbf{Total Demand} is the sum of current imported and domestic oil consumed. Since the EIA reports the share of total consumption that is domestically-sourced, demand is calculated as: $$\text{Total Demand}_i = \sum_j e_{j,i} \cdot (1 - \text{Domestic Share})^{-1}$$
	With $Total Imports$ set at the initial baseline. If current imports increase (as explained below) then supply will exceed demand. 
	\item \textbf{Total Exports} is the sum of current exported oil to all other countries.
\end{itemize}

\paragraph{Crisis Submodel}
\begin{itemize}
	\item \textbf{Instability} is a measure of the country's political instability on a 0-10 scale, and determines the probability of a crisis or conflict -- a production shock. It is obtained from \citet{eiu_2013}. 
	\item \textbf{In Crisis?} is simply a boolean variable determining whether or not a country is currently experiencing a crisis or conflict.
	\item \textbf{Crisis Length} is the number of months remaining in a crisis, if one is currently in process. When a country enters crisis, this number is drawn from a power law distribution, as described below.
\end{itemize}

\paragraph{Assistance Submodel}
\begin{itemize}
	\item \textbf{Increasing Production} is a boolean of whether or not the country is currently in a state of temporarily-increased production.
	\item \textbf{Increasing Production For} stores the country being assisted. I assume that countries target their assistance at one other country.
		\item \textbf{Crisis Length} is the number of months remaining in a crisis, if one is currently in process. When a country enters crisis, this number is drawn from a power law distribution, as described below.
			\item \textbf{Total Capacity} is the country's capacity to rapidly increase oil for export above its inital baseline. Defaults to 1.1 (that is, countries can rapidly increase output by 10\%) based on consultations with subject-matter experts. 
\end{itemize}

\subsubsection{Agent Behavior}

Each tick of the model, agents are activated in random order, as follows:

\begin{algorithm}[H]
	\caption{main loop}
	\eIf{not in crisis} {
			Call crisis submodel \;
		}{
			$Crisis Length = Crisis Length - 1$ \;
			\If {Crisis Length == 0} { End crisis \; }
		}
	\If{Assistance turned on} {
		Call assistance submodel \;
	}	
\end{algorithm}

\begin{algorithm}[H]
	\caption{crisis submodel}
		enter crisis with $Pr(crisis) \propto instability$ \;
		\If{entered crisis}{
			Crisis Length $\mathtt{\sim}  \text{Power Law}$ \;
			\If{Contagion turned on} {
				\For{each neighboring country} {
					Call neighbor's crisis submodel \;
			}}}
\end{algorithm}


\begin{algorithm}[H]
	\caption{assistance submodel}
	\eIf{Currently assisting another country} {
			\If{Assisted country no longer in supply shock} {
				\tcp{Ending assistance}
				\For{each export edge} {
					edge volume = baseline edge volume;\
				}
			}
		} {
			\For{each out-neighbor} {
				\If{Neighbor in supply shock} {
					Assist neighbor with $Pr = \frac{\text{Exports to neighbor}}{\text{Total exports}}$
				}
			}
			\If{Assisting neighbor} {
				\tcp{Increasing output}
				\For{each export edge} {
					edge volume = edge volume * (1 + excess capacity)\;
				}
			}
		}

\end{algorithm}

Once all agents have acted, their energy-security metrics are computed, as described below.

\subsubsection{Energy Security Metrics}

Let $\delta(i)$ be the \emph{Security indicator} function, defined as:
\[
\delta(i) = \begin{dcases*}
	1 & when $i$ is not in crisis\\
	0 & when $i$ is in crisis
\end{dcases*}
\]

Country-level indicators are:

\begin{itemize}
	\item \textbf{Supply Ratio} is the fraction of the country's current demand being met by \emph{secure} imports -- that is, imports from countries that are not in crisis. Formally:
	$$
		\text{Supply Ratio}_i = \frac{\text{Total Demand}_i}{\sum_{j}e_{j,i}\delta(j)}
	$$
\item \textbf{Demand Ratio} is the fraction of the country's current exports going to countries that are not in crisis -- secure exports. 
$$
\text{Demand Ratio}_i = \frac{\sum_{j}e_{i,j}\delta(j)}{\sum_{j}e_{i,j}}
$$
\end{itemize}

The model also computes similar global ratios:
\begin{itemize}
	\item \textbf{Global Supply Ratio} is the all country's current demand being met by secure imports:
	$$
		\text{Global Supply Ratio} = \frac{\sum_i\text{Total Demand}_i}{\sum_i\sum_{j}e_{j,i}\delta(j)}
	$$
	\item \textbf{Global Demand Ratio} is the fraction of worldwide exports going to countries that are not in crisis:
	$$
		\text{Global Demand Ratio} = \frac{\sum_i \text{Total Demand}_i\delta(i)}{\sum_i\sum_{j}e_{i,j}}
	$$
	\item \textbf{Global Overall Ratio} is the ratio between secure supply and secure demand:
	$$
		\text{Global Overall Ratio} = \frac{\sum_i \text{Total Demand}_i\delta(i)}{\sum_i\sum_{j}e_{j,i}\delta(j)}
	$$
\end{itemize}


\section{Model Results and Analysis}

\subsection{Validation}

\subsection{Global Outcome Range}

I run the model for 100 iterations for each permutation of the Contagion and Assistance flags, 400 iterations in total; each iteration was run for 60 ticks, or 5 simulated years. I collect the country-level and global indicators for each month of each iteration. 

Figure X shows a histogram of the Global Overall Ratio (the ratio between secure demand and supply) across all iteration-months. It appears to approximate a skew normal distribution, centered on 1 (a balance between supply and demand) but with a longer tail to the right -- supply shocks are more likely than demand shocks, and the most extreme supply shocks the model generates, though low-probability, are more extreme than any demand shocks. 

We can also examine the model iterations separately. Initially, we can characterize each iteration by the mean and variance of each indicator, as shown in Figure X. A high Overall or Supply mean ratio may indicate overall higher oil insecurity, but if the variance is low, it is `stably' insecure; in contrast, the mean may be close to 1 or even below it, but high variance would indicate severe uncertainty and fluctuation -- a distinct type of insecurity. Note that the overall correlation between mean and variance is XX, indicating that they are not directly linked to one another.

We can further look at the trace of the indicators within each iteration, as they change from tick to tick. Figure X shows the traces of the 400 model runs overlaid on one another, and indicates some clear patterns. The majority of runs stay within a fairly narrow band, indicating similar levels of stability. Spikes outside the band, when they occur, tend to be brief. Note, however, that there are some longer-term spikes -- shocks which persist for multiple months, yielding values that become at least temporary baselines, even seeing fluctuations around them.

\subsection{Contagion and Assistance}

In order to disaggregate the effects of the Contagion and Assistance model parameters, we examine their results separately. Figure XX shows histograms of the scenario-months for Contagion and Assistance, respectively. The Contagion comparison indicates that conflict contagion does not appear to change the mode of the ratio distribution, which for both remains very close to 1. However, contagion does increase the variance of the distribution, making extreme outcomes more likely -- both to the right, indicating supply shocks, but also to the left, indicating demand shocks. 

Similarly, the modes of the distributions of ratios for scenario-months where Assistance is set to on and off are similarly both close to 1. 

\section{Discussion}



In this poster, I will present an initial implementation of a novel agent-based model linking crude oil energy security to inter- and intra-state crises and conflicts, through the mechanism of supply shocks \citep{kilian_2008}. The model operates at the meso (months-years) timescale, and allows countries to react to changes in the oil supply. It can generate ranges of notional near-future trajectories of global and local energy security, providing risk estimates for different countries. The model attempts to capture the emergent consequences of conflict contagion \citep{black_2013}, temporary production increases and varying levels of stockpiles. It can also be used to explore different scenarios, from particular crises to realignments of major trade relationships. 

The agents in the model are geospatially located countries, acting in monthly timesteps. Countries experience exogenous crises based on probabilities derived from empirical estimates (such as \citep{goldstone_2005,eiu_2013,ward_2013}). They are linked geographically, and by oil trade relationships obtained via the COMTRADE database \citep{un_2013}. As countries experience crises, their participation in the  international oil system is placed at risk, changing the volume of `secure' imports or exports of other countries and  potentially triggering supply or demand shocks. 

Countries respond to the changing energy security environment in several ways. Exporters may choose to temporarily increase production in response to supply shocks elsewhere, while countries that are members of the International Energy Agency may vote to release their emergency stockpiles. The model's behavior is verified against previous qualitative and quantitative analysis, and in consultation with subject-matter experts. 

The poster will describe the model's methodology and input data, and provide a visual representation of the model's estimated energy security risk for every country. It will provide a visual example of a representative model run, and illustrate the results of a sweep of the model's behavior and parameter space. Finally, it will describe future work, incorporating different energy sources, shipping chokepoints, and dynamic network updating.

 \bibliographystyle{chicago}
\bibliography{energysecurity}

\end{document}