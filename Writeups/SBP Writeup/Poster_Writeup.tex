\documentclass{llncs}

\begin{document}

\title{Simulating International Energy Security}
	\author{David Masad}
	\institute{Department of Computational Social Science \linebreak George Mason University \linebreak Fairfax, Virginia, USA}
	\maketitle


Energy security is defined by the International Energy Agency as ``the uninterrupted availability of energy sources at an affordable price''\cite{iea_2013}. As the world's energy demand continues to expand in the face of growing scarcity, energy security is becoming increasingly critical not just to developed countries but to developing ones as well \cite{yergin_2006}. Thus, assessing the energy security of different countries is important in order to understand the overall state of the international system.

Much of the current analysis of energy security focuses either on short-term vulnerabilities (in the time-scale of days and weeks \cite{Jewell_2011}), or on long-term investments in new technologies over the course of years and decades \cite{jacobson_2009}. The short-term analyses have largely concentrated on static indicators \cite{Jewell_2011}, with a particular focus on portfolio diversity (e.g. \cite{wu_2009,skea_2010,stirling_2010}), and do not incorporate the complex interactions that characterize the international system (\cite{cederman_1997,geller_2011}).

In this poster, I will present an initial implementation of a novel agent-based model linking crude oil energy security to inter- and intra-state crises and conflicts, through the mechanism of supply shocks \cite{kilian_2008}. The model operates at the meso (months-years) timescale, and allows countries to react to changes in the oil supply. It can generate ranges of notional near-future trajectories of global and local energy security, providing risk estimates for different countries. The model attempts to capture the emergent consequences of conflict contagion \cite{black_2013}, temporary production increases and varying levels of stockpiles. It can also be used to explore different scenarios, from particular crises to realignments of major trade relationships. 

The agents in the model are geospatially located countries, acting in monthly timesteps. Countries experience exogenous crises based on probabilities derived from empirical estimates (such as \cite{goldstone_2005,eiu_2013,ward_2013}). They are linked geographically, and by oil trade relationships obtained via the COMTRADE database \cite{un_2013}. As countries experience crises, their participation in the  international oil system is placed at risk, changing the volume of `secure' imports or exports of other countries and  potentially triggering supply or demand shocks. 

Countries respond to the changing energy security environment in several ways. Exporters may choose to temporarily increase production in response to supply shocks elsewhere, while countries that are members of the International Energy Agency may vote to release their emergency stockpiles. The model's behavior is verified against previous qualitative and quantitative analysis, and in consultation with subject-matter experts. 

The poster will describe the model's methodology and input data, and provide a visual representation of the model's estimated energy security risk for every country. It will provide a visual example of a representative model run, and illustrate the results of a sweep of the model's behavior and parameter space. Finally, it will describe future work, incorporating different energy sources, shipping chokepoints, and dynamic network updating.

\bibliographystyle{splncs}
\bibliography{sbp_bib}

\end{document}